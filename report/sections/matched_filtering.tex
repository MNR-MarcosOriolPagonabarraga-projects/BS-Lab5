\section{Matched Filtering}

\subsection{Impulse Response Derivation}
The matched filter is the optimal linear filter for maximizing the Signal-to-Noise Ratio (SNR) in the presence of stochastic noise. The impulse response $h(n)$ of a matched filter is the time-reversed version of the template signal $s(n)$.

Figure \ref{fig:impulse_resp} displays the template extracted in the previous section and the derived impulse response. As expected, $h(n)$ is a mirrored version of $s(n)$.

\begin{figure}[H]
    \centering
    \includegraphics[width=0.9\textwidth]{img/sec3_impulse_response.png}
    \caption{Top: The SWC template $s(n)$. Bottom: The matched filter impulse response $h(n) = s(-n)$.}
    \label{fig:impulse_resp}
\end{figure}

\subsection{Linear Matched Filtering (Unnormalized)}
We first applied the matched filter using the standard linear convolution method. The filter coefficients were defined as the reversed template ($b = h(n)$) and the denominator $a=1$.

\begin{lstlisting}[caption={Linear Matched Filtering Implementation}, language=Matlab]
% Derive Impulse Response
h_t = fliplr(template); 

% Apply FIR Filter
matched_output = filter(h_t, 1, signal);
\end{lstlisting}

Figure \ref{fig:linear_c3} shows the result for channel C3. While the filter successfully highlights the periodic nature of the SWCs, the output amplitude is extremely high (in the order of $10^8$). This occurs because the standard convolution accumulates the product of the signal and template energies (effectively $K=1$). This unbounded output makes it difficult to set a universal detection threshold.

\begin{figure}[H]
    \centering
    \includegraphics[width=0.9\textwidth]{img/sec3_matched_linear_c3.png}
    \caption{Unnormalized matched filter output for C3. Note the large amplitude scale.}
    \label{fig:linear_c3}
\end{figure}

\subsection{Normalized Matched Filtering (Time-Variant K)}
To bound the output between -1 and 1 (similar to the normalized cross-correlation), we implemented a normalization factor $K$. Since the energy of the EEG signal changes over time, $K$ must be \textbf{time-variant}.

The normalization factor at each sample is derived from the geometric mean of the template energy ($E_{template}$) and the local signal energy ($E_{local}$).

\begin{lstlisting}[caption={Normalized Filtering with Time-Variant K}, language=Matlab]
% Calculate Template Energy
E_template = sum(template.^2);

% Calculate Local Signal Energy
ones_filter = ones(size(template));
E_local = filter(ones_filter, 1, signal.^2);

% Apply Normalization
normalization_factor = sqrt(E_template * E_local);
y_normalized = y_raw ./ normalization_factor;
\end{lstlisting}

\subsection{Results and Discussion}
The normalized matched filter was applied to channels C3, C4, and P4. A threshold of 0.2 was used to detect the Spike-and-Wave Complexes.

\begin{figure}[H]
    \centering
    \includegraphics[width=0.9\textwidth]{img/sec3_matched_normalized_c3.png}
    \caption{Normalized Matched Filter output for Channel C3.}
    \label{fig:norm_c3}
\end{figure}

\begin{figure}[H]
    \centering
    \includegraphics[width=0.9\textwidth]{img/sec3_matched_normalized_c4.png}
    \caption{Normalized Matched Filter output for Channel C4.}
    \label{fig:norm_c4}
\end{figure}

\begin{figure}[H]
    \centering
    \includegraphics[width=0.9\textwidth]{img/sec3_matched_normalized_p4.png}
    \caption{Normalized Matched Filter output for Channel P4.}
    \label{fig:norm_p4}
\end{figure}

\subsubsection{Analysis}
The results demonstrate the equivalence between \textbf{normalized cross-correlation} (Section 2) and \textbf{normalized matched filtering} (Section 3).
\begin{itemize}
    \item \textbf{C3 (Source):} Figure \ref{fig:norm_c3} shows a perfect correlation of 1.0 at the template extraction point. The detected peaks align perfectly with the epileptic complexes.
    \item \textbf{C4 (Close):} As seen in Figure \ref{fig:norm_c4}, the matched filter is highly robust, detecting SWCs with correlation values consistently above 0.5.
    \item \textbf{P4 (Far):} Figure \ref{fig:norm_p4} shows that even in the parietal region, where the signal morphology differs slightly, the matched filter successfully identifies the events.
\end{itemize}

The matched filter approach, when properly normalized, provides a robust detection mechanism. The primary difference between this and the method in Section 2 is computational: Section 2 used `xcorr` (global computation), while Section 3 used `filter` (causal convolution), which is more suitable for real-time processing applications.